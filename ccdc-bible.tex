\documentclass[
    11pt,
    titlepage
]{extarticle}

\usepackage{fontspec}

\setmainfont{Noto Serif}

\setsansfont{Noto Sans}

\setmonofont{Noto Sans Mono}

\usepackage[
    margin=1in
]{geometry}

\input{build/tex-preamble.tex}
\providecommand{\version}{}%
\providecommand{\builddate}{\today}%


\title{CCDC Bible}
\author{CCDC Team Blood, Sweat, and Tears}
\date{\today}

\begin{document}
    \maketitle
    \tableofcontents
    \clearpage
    \section{Contributors}\label{contributors}

    \subsection{2025}\label{2025}

    \subsubsection{Turner Bushell - Team
    Leader}\label{turner-bushell---team-leader}

    \subsubsection{Ethan Hunt - Linux/Web}\label{ethan-hunt---linuxweb}

    \subsubsection{Hayden Robbins -
    Windows}\label{hayden-robbins---windows}

    \subsubsection{Caleb Davis -
    Networking}\label{caleb-davis---networking}

    \subsubsection{Jarom Smith - Windows}\label{jarom-smith---windows}

    \subsubsection{Crystal Hammond -
    Injects}\label{crystal-hammond---injects}

    \subsubsection{Mckay Fawcett -
    Injects}\label{mckay-fawcett---injects}

    \subsubsection{Spring Bitner -
    Injects}\label{spring-bitner---injects}

    \subsubsection{Lucas Bigler - Backup}\label{lucas-bigler---backup}

    \section{First 30 Minutes}\label{first-30-minutes}

    \subsection{Windows}\label{windows}

    \subsubsection{1. Run Hayden's Script}\label{run-haydens-script}

    \begin{itemize}
    \item
      Quick Install:

\begin{lstlisting}
[Net.ServicePointManager]::SecurityProtocol = [Net.SecurityProtocolType]::Tls12;
Invoke-WebRequest https://github.com/SUU-Cybersecurity-Club/CCDC-Scripts/releases/latest/download/windows-hardening.zip -Outfile windows-hardening.zip;
Expand-Archive -Path windows-hardening.zip -DestinationPath windows-hardening;
\end{lstlisting}
    \item
      Run

\begin{lstlisting}
cd windows-hardening;
.\start.bat;
\end{lstlisting}
    \end{itemize}

    \subsubsection{2. Take Screenshot of
    login}\label{take-screenshot-of-login}

    \begin{itemize}
    \item
      Note service versions
    \item
      Nmap scan
    \end{itemize}

    \subsubsection{3. Disable or change passwords to all unneeded
    users}\label{disable-or-change-passwords-to-all-unneeded-users}

    \subsection{\texorpdfstring{\textbf{Linux}}{Linux}}\label{linux}

    \subsubsection{1. Password Change, Check
    Sudoers}\label{password-change-check-sudoers}

    \begin{itemize}
    \item
      Run

\begin{lstlisting}
passwd
\end{lstlisting}
    \item
      Add a new backup user, and add them to sudo group

\begin{lstlisting}
useradd <username>
passwd <username>
sudo usermod -aG sudo
\end{lstlisting}
    \item
      Check sudoers

\begin{lstlisting}
visudo
/etc/sudoers
/etc/sudoers.d
\end{lstlisting}
    \end{itemize}

    Confirm from another terminal that updated password and backup user
    works

    \subsubsection{2. Backup scored
    services}\label{backup-scored-services}

    \textbf{Backup \passthrough{\lstinline!/etc!} every time}

    \begin{itemize}
    \item
      Backup Syntax

\begin{lstlisting}
tar -cf <new_file_name> <thing getting backed up>
cd /
tar -cf ettc etc
mv ettc <someplace/in excel>
\end{lstlisting}
    \end{itemize}

    Remember to check \passthrough{\lstinline!/var/www/html!}

    \begin{itemize}
    \item
      Database

\begin{lstlisting}
mysql -u <username> -p -e "SHOW DATABASES; EXIT;"
mysqldump -u <username> -p <database_name> > <database_name>_backup.sql
mysqldump -u <username> -p <database_name table_name> > <table_name>_backup.sql
mysqldump -u <username> -p <database_name table{num}> > backup.sql
\end{lstlisting}
    \end{itemize}

    \subsubsection{3. Remove SSH if not scored, if needed backup keys,
    then
    remove}\label{remove-ssh-if-not-scored-if-needed-backup-keys-then-remove}

    \begin{itemize}
    \item
      Ubuntu/Debian

\begin{lstlisting}
apt remove openssh-server
\end{lstlisting}
    \item
      Fedora/CentOS

\begin{lstlisting}
yum erase openssh-server
dnf remove openssh-server
\end{lstlisting}
    \end{itemize}

    \subsubsection{4. Login Banner}\label{login-banner}

    \begin{itemize}
    \item
      Edit \passthrough{\lstinline!/etc/ssh/sshd\_config!}

\begin{lstlisting}
banner /etc/issue.net
\end{lstlisting}
    \item
      Edit or create \passthrough{\lstinline!/etc/issue.net!}

      \begin{itemize}
      \tightlist
      \item
        Place banner text
      \end{itemize}
    \item
      Restart \passthrough{\lstinline!sshd!} if needed

\begin{lstlisting}
systemctl restart sshd
\end{lstlisting}
    \item
      Take a screenshot of banner
    \end{itemize}

    \subsubsection{5. Install Wazuh Agent}\label{install-wazuh-agent}

    \begin{itemize}
    \item
      Debian/Ubuntu

\begin{lstlisting}
wget https://packages.wazuh.com/4.x/apt/pool/main/w/wazuh-agent/wazuh-agent_4.10.1-1_amd64.deb
sudo WAZUH_MANAGER='172.20.241.20' dpkg -i ./wazuh-agent_4.10.1-1_amd64.deb
sudo systemctl daemon-reload
sudo systemctl enable --now wazuh-agent
\end{lstlisting}
    \item
      CentOS/Fedora

\begin{lstlisting}
curl -o wazuh-agent.rpm
https://packages.wazuh.com/4.x/yum/wazuh-agent-4.10.1-1.x86_64.rpm
sudo WAZUH_MANAGER='172.20.241.20' WAZUH_AGENT_NAME='Fedora' rpm -ihv wazuh-agent.rpm
sudo systemctl daemon-reload; sudo systemctl enable --now wazuh-agent
\end{lstlisting}
    \end{itemize}

    \subsubsection{6. Remove RevShells - Cron, Bad/Faulty
    Services}\label{remove-revshells---cron-badfaulty-services}

    \begin{itemize}
    \item
      List crontabs

\begin{lstlisting}
crontab -l
crontab -l -u <user>
\end{lstlisting}
    \item
      Edit crontabs

\begin{lstlisting}
crontab -e
crontab -e -u <user>
\end{lstlisting}
    \end{itemize}

    \subsubsection{7. Backup any other important
    services}\label{backup-any-other-important-services}

    \begin{itemize}
    \tightlist
    \item
      Figure out what that is there
    \end{itemize}

    \subsubsection{8. Run nmap scan}\label{run-nmap-scan}

    \begin{itemize}
    \item
      Run scan, and save to file

\begin{lstlisting}
nmap -sV -T4 -p- localhost > nmap.txt
\end{lstlisting}
    \end{itemize}

    \subsubsection{9. Firewall}\label{firewall}

    \begin{itemize}
    \item
      CentOS/Fedora (\passthrough{\lstinline!firewall-cmd!})

\begin{lstlisting}
firewall-cmd --state
firewall-cmd --list-all-zones
firewall-cmd --set-target=DROP --permanent
firewall-cmd --add-port=80/tcp --permanent
firewall-cmd --add-service=http --permanent
firewall-cmd --add-rich-rule='rule family="ipv4" destination port=25 protocol=tcp reject' --permanent
firewall-cmd --reload
\end{lstlisting}
    \item
      Ubuntu \textless{} 20.04, Debian \textless{} 10
      (\passthrough{\lstinline!ufw!})

\begin{lstlisting}
ufw status
ufw status verbose
ufw default deny incoming
ufw allow 80
ufw allow 80/tcp
ufw deny out 25/tcp
ufw enable
\end{lstlisting}
    \item
      Ubuntu \textgreater= 20.04, Debian \textgreater= 10, Fedora
      \textgreater= 32, RHEL \textgreater= 8
      (\passthrough{\lstinline!nftables!})

\begin{lstlisting}
nft list ruleset
nft add table inet filter
nft add chain inet filter input { type filter hook input priority 0 \; policy drop \; }
nft add chain inet filter output { type filter hook output priority 0 \; policy accept \; }
nft add rule inet filter input tcp dport 80 accept
nft add rule inet filter output tcp dport 25 drop
nft list ruleset > /etc/nftables.conf
\end{lstlisting}
    \item
      Generic Linux (\passthrough{\lstinline!iptables!})

\begin{lstlisting}
iptables -P INPUT DROP
iptables -A INPUT -p tcp --dport 80 -j ACCEPT
iptables -A OUTPUT -p tcp --dport 25 -j REJECT
apt-get install iptables-persistent
netfilter-persistent save
systemctl enable netfilter-persistent
iptables -S
iptables -L
iptables -L -v
iptables -t filter -L -v
iptables -t nat -L -v
iptables -t mangle -L -v
iptables -t raw -L -v
iptables -t security -L -v
\end{lstlisting}
    \end{itemize}

    \subsubsection{10. Backup Again}\label{backup-again}

    \begin{itemize}
    \item
      Backup \passthrough{\lstinline!/etc!}

\begin{lstlisting}
tar -cf <backup_name> /etc
\end{lstlisting}
    \end{itemize}

    \subsubsection{Scripts or Steps}\label{scripts-or-steps}

    \begin{itemize}
    \item
      Download scripts

\begin{lstlisting}
curl -O https://github.com/SUU-Cybersecurity-Club/CCDC-Scripts/releases/latest/linux-hardening.tar.xz
\end{lstlisting}
    \item
      Extract archive and enter directory

\begin{lstlisting}
tar -xpf linux-hardening.tar.xz
cd linux-hardening
\end{lstlisting}
    \item
      Mark executable

\begin{lstlisting}
chmod +x start.sh linux_wazuh_agent.sh
\end{lstlisting}
    \item
      Run Scripts

\begin{lstlisting}
./start.sh
\end{lstlisting}
    \end{itemize}

    \subsubsection{Wazuh}\label{wazuh}

    \begin{itemize}
    \item
      Setup Wazuh Server on Splunk/Unused machine
    \item
      Create Groups for each scored service
    \item
      Setup Conditional Filters
    \item
      Create secondary admin user
    \item
      File Integrity Monitoring
    \end{itemize}

    \subsubsection{Firewall}\label{firewall-1}

    \begin{itemize}
    \item
      Change Admin Password
    \item
      Remove other Admins
    \item
      Fix Weird Firewall Rules
    \item
      Configure Objects
    \item
      Backup Firewall
    \item
      Note Service Versions
    \item
      TAKE A Screenshot
    \end{itemize}

    \subsection{\texorpdfstring{\textbf{Palo
    Alto}}{Palo Alto}}\label{palo}

    \subsubsection{1. Create new users and delete old
    admins}\label{create-new-users-and-delete-old-admins}

    \subsubsection{2. Check admin roles and delete unnecessary
    ones}\label{check-admin-roles-and-delete-unnecessary-ones}

    \subsubsection{3. Check Global protect settings under
    network}\label{check-global-protect-settings-under-network}

    \subsubsection{4. Perform updates}\label{perform-updates}

    \begin{itemize}
    \item
      Download 11.1.0 (software page) and new content version (dynamic
      page) refresh to show
    \item
      Download 11.1.4-h7 (or current preferred version) and current
      antivirus (dynamic page)
    \item
      This should take about 20 min with a 4 min lapse in connection for
      each install
    \end{itemize}

    \subsubsection{5. Start making rules}\label{start-making-rules}

    \section{General Stuff}\label{general-stuff}

    \subsection{\texorpdfstring{\textbf{PEMDAS}}{PEMDAS}}\label{pemdas}

    \subsubsection{\texorpdfstring{\textbf{P}atch \textbf{E}verything,
    \textbf{M}onitor \textbf{D}ata, \textbf{A}uthenticate
    \textbf{S}ecurely}{Patch Everything, Monitor Data, Authenticate Securely}}\label{patch-everything-monitor-data-authenticate-securely}

    \begin{enumerate}
    \def\labelenumi{\arabic{enumi}.}
    \item
      Password changes
    \item
      Backup services (etc, var, sql databases.)
    \item
      Patch immediate vulnerabilities (remove bad users, clear crontabs,
      uninstall unnecessary services)
    \item
      Add redundant users
    \item
      Scan network from win 10 for services that shouldnt be open
    \item
      Profit
    \end{enumerate}

    \subsection{\texorpdfstring{\textbf{Linux}}{Linux}}\label{linux}

    \subsubsection{Account/Group}\label{accountgroup}

    \begin{itemize}
    \item
      Add user

\begin{lstlisting}
useradd -m -G <groups> -s /bin/ <username>
\end{lstlisting}
    \item
      Add user to sudo group

\begin{lstlisting}
sudo usermod -aG sudo <username>
\end{lstlisting}
    \item
      Change password (leave \passthrough{\lstinline!<username>!} blank
      to change password for current user)

\begin{lstlisting}
passwd <username>
\end{lstlisting}
    \end{itemize}

    \subsubsection{Archives}\label{archives}

    \begin{itemize}
    \item
      Make tarball

\begin{lstlisting}
tar -cf <filename>.tar <file/dir to archive>
\end{lstlisting}
    \item
      List tarball contents

\begin{lstlisting}
tar -tf <filename>.tar
\end{lstlisting}
    \item
      Unpack tarball

\begin{lstlisting}
tar -xf <filename>.tar
\end{lstlisting}
    \item
      Unpack \passthrough{\lstinline!/etc/passwd!}

\begin{lstlisting}
tar -xf <filename>.tar --strip-components=2 etc/passwd
\end{lstlisting}
    \end{itemize}

    \begin{tipbox}{Tar Flags}

    Most tar flags are pretty self explanitory, like
    \passthrough{\lstinline!c!} for compress and
    \passthrough{\lstinline!x!} for extract

    \end{tipbox}

    \subsubsection{Database}\label{database}

    \begin{itemize}
    \item
      Change root password

\begin{lstlisting}
mysql -u root -p -e "USE mysql; ALTER USER 'root'@'%' IDENTIFIED BY 'new_secure_password'; FLUSH PRIVILEGES; SHOW TABLES; EXIT;"
\end{lstlisting}
    \item
      Backup

\begin{lstlisting}
mysqldump -u <username> -p <database_name table_name> > <table_name>_backup.sql
mysqldump -u <username> -p <database_name table{num}> > backup.sql
\end{lstlisting}
    \item
      Show Users

\begin{lstlisting}
mysql -u root -p -e "SELECT User, Host FROM mysql.user;"
\end{lstlisting}
    \item
      Restore Backup

\begin{lstlisting}
mysql -u <username> -p <database_name> < <database_name>_backup.sql
\end{lstlisting}
    \item
      Overwrite Backup

\begin{lstlisting}
mysql -u <username> -p -e "DROP DATABASE IF EXISTS <database_name>; CREATE DATABASE <database_name>;"
mysql -u <username> -p <database_name> < <database_name>_backup.sql
\end{lstlisting}
    \item
      Restore Table Backup

\begin{lstlisting}
mysql -u <username> -p <database_name> < <table_name>_backup.sql
\end{lstlisting}
    \item
      Overwrite Table Backup

\begin{lstlisting}
mysql -u <username> -p -e "USE <database_name>; DROP TABLE IF EXISTS <table_name>;"
mysql -u <username> -p <database_name> < <table_name>_backup.sql
\end{lstlisting}
    \end{itemize}

    \subsubsection{Processes}\label{processes}

    \begin{itemize}
    \item
      List process as forest

\begin{lstlisting}
ps -eaf --forest
\end{lstlisting}
    \item
      List processes using ports

\begin{lstlisting}
ss -autpn
\end{lstlisting}
    \item
      Check where process is running from

\begin{lstlisting}
cd /proc/<pid>
ls -la | grep cwd
\end{lstlisting}
    \item
      Show socket information from pid

\begin{lstlisting}
ss -anp | grep <pid>
\end{lstlisting}
    \item
      Force kill process (SIGKILL)

\begin{lstlisting}
kill -9 <pid>
\end{lstlisting}
    \item
      Graceful kill process (SIGTERM)

\begin{lstlisting}
kill -15 <pid>
\end{lstlisting}
    \end{itemize}

    \subsubsection{Files}\label{files}

    \begin{itemize}
    \item
      Check for NFS and SMB shares on host

\begin{lstlisting}
nmap --script="nfs-showmount,smb-os-discovery" -p 2049,445 localhost
\end{lstlisting}
    \item
      Open multiple files in directory

\begin{lstlisting}
find <dir_path> -type f -exec cat {} + | less
\end{lstlisting}
    \end{itemize}

    \subsubsection{LinPeas}\label{linpeas}

    \begin{itemize}
    \item
      Increase readability

\begin{lstlisting}
./linpeas.sh | less -RN
\end{lstlisting}
    \end{itemize}

    \subsection{\texorpdfstring{\textbf{Windows}}{Windows}}\label{windows}

    \subsubsection{2016 Docker/Remote}\label{dockerremote}

    \subsubsection{\texorpdfstring{\textbf{Run Hayden's
    script}}{Run Hayden's script}}\label{run-haydens-script-1}

    If you're lame and don't run Hayden's script follow these
    instructions

    \begin{enumerate}
    \def\labelenumi{\arabic{enumi}.}
    \item
      Change passwords
    \item
      Disable all unnecessary accounts (If the accounts are required
      because score goes down make sure to reenable them then remove
      them from all groups and resources and give them no permissions.)
    \item
      Create new admin user for team use
    \item
      Remove SMB 1
    \item
      Disable RDP - ServerManager/Local Server/Remote Desktop Disabled
    \item
      Fix the DNS on the Machine (IPV4 Driver)
    \item
      Setup the Powershell Connection Script (From now on you will watch
      this periodically to see if Redteam is in.)
    \item
      Install Wazuh Agent on the machine.
    \item
      Turn on Firewall
    \item
      Ensure that Defender Antivirus is Enabled
    \item
      Open Services, Disable the Printer Spooler Service and Stop it
    \item
      Start Windows Update
    \end{enumerate}

    \textbf{--Once you're done the above, continue below--}

    \begin{enumerate}
    \def\labelenumi{\arabic{enumi}.}
    \setcounter{enumi}{12}
    \item
      Check the SMB Shares for bad setup (Like the entire C: being
      shared) Check Share Permissions.
    \item
      Check for Unnecessary Services/Programs and uninstall.
    \item
      Save Process and Services list to XML so that you can check if
      things change later.
    \item
      Check Firewall Rules
    \item
      Setup Good Group Policy.
    \end{enumerate}

    \textbf{--Injects information--}

    \begin{enumerate}
    \def\labelenumi{\arabic{enumi}.}
    \item
      Setup a Warning Message for those who login (Group Policy Object)
    \item
      Run an NMAP against the box and take a screenshot (Preferably from
      a different machine, Zenmap include netcat capabilities which you
      don't need. use Windows 10)
    \end{enumerate}

    \begin{itemize}
    \item
      Check for remote connections

\begin{lstlisting}
where(1) { Get-NetTCPConnection -State Established | Select LocalAddress,LocalPort, RemoteAddress,RemotePort,OwningProcess,@{Name="cmdline";Expression={(Get-WmiObject Win32_Process -filter "ProcessId” = $($_.OwningProcess)).commandline} | Where-Object {$_.RemoteAddress -NE "127.0.0.1" -AND $_.RemoteAddress -NE "127.0.0.0” -AND $_.RemoteAddress -NE "::" -AND $_.RemoteAddress -NE "::1" -AND $_.RemoteAddress -NE "172.20.240.10" -AND $_.RemotePort -NE "80" -AND $_.RemotePort -NE "443"}; sleep 5; clear;}
\end{lstlisting}
    \end{itemize}

    \subsubsection{Hardening Script}\label{hardening-script}

    \begin{itemize}
    \item
      Located at
      https://github.com/SUU-Cybersecurity-Club/CCDC-Scripts/tree/main/windows-hardening
    \item
      README has instructions for running the script
    \item
      Installation instructions

\begin{lstlisting}
[Net.ServicePointManager]::SecurityProtocol = [Net.SecurityProtocolType]::Tls12;
Invoke-WebRequest https://github.com/SUU-Cybersecurity-Club/CCDC-Scripts/releases/latest/download/windows-hardening.zip -Outfile windows-hardening.zip;
Expand-Archive -Path windows-hardening.zip -DestinationPath windows-hardening;
\end{lstlisting}
    \item
      Run Script

\begin{lstlisting}
cd windows-hardening;
.\start.bat;
\end{lstlisting}
    \end{itemize}

    \section{Operating Systems}\label{operating-systems}

    \subsection{\texorpdfstring{\textbf{Debian
    10}}{Debian 10}}\label{debian10}

    \subsubsection{Network}\label{network}

    \begin{itemize}
    \item
      DNS: Should be \passthrough{\lstinline!8.8.8.8!}
    \item
      Change apt repos to pull from archive mirror

\begin{lstlisting}
sed -i 's/deb.debian.org/archive.debian.org/g' /etc/apt/sources.list 
sed -i 's/security.debian.org/archive.debian.org/g' /etc/apt/sources.list 
sed -i '/stretch-updates/d' /etc/apt/sources.list
\end{lstlisting}
    \end{itemize}

    \subsubsection{User Admin}\label{user-admin}

    \begin{itemize}
    \item
      Change passwords

\begin{lstlisting}
passwd
passwd sysadmin
passwd produdu
\end{lstlisting}
    \item
      Add backup user

\begin{lstlisting}
useradd <backup_username>
password <backup_password>
\end{lstlisting}
    \item
      Add backup user to sudo group

\begin{lstlisting}
usermod -aG sudo <backup_username>
\end{lstlisting}
    \end{itemize}

    \subsubsection{Backup 1}\label{backup-1}

    \begin{itemize}
    \item
      Backup \passthrough{\lstinline!/etc!}

\begin{lstlisting}
cd /
tar -cf bcte etc/
\end{lstlisting}
    \end{itemize}

    \subsubsection{Script}\label{script}

    \begin{itemize}
    \item
      Download scripts

\begin{lstlisting}
curl -O https://github.com/SUU-Cybersecurity-Club/CCDC-Scripts/releases/latest/linux-hardening.tar.xz
\end{lstlisting}
    \item
      Extract archive and enter directory

\begin{lstlisting}
tar -xpf linux-hardening.tar.xz
cd linux-hardening
\end{lstlisting}
    \item
      Mark executable

\begin{lstlisting}
chmod +x start.sh linux_wazuh_agent.sh
\end{lstlisting}
    \item
      Run Scripts

\begin{lstlisting}
./start.sh
\end{lstlisting}
    \end{itemize}

    \subsubsection{Services}\label{services}

    \begin{itemize}
    \item
      Nuke SSH

\begin{lstlisting}
apt remove openssh-server
\end{lstlisting}
    \item
      Disable vulnerable services

\begin{lstlisting}
systemctl stop sshd
systemctl stop exim4
systemctl disable exim4
systemctl stop avahi-daemon
systemctl disable avahi-daemon
systemctl stop minissdpd
systemctl disable minissdpd
systemctl stop proftpd
systemctl disable proftpd
systemctl stop apache2
systemctl disable apache2
systemctl stop rpcbind
systemctl disable rpcbind
systemctl stop ntpd
systemctl disable nptd
\end{lstlisting}
    \item
      Remove packages for services we don't want

\begin{lstlisting}
apt remove apache2 proftpd-basic postfix nginx
\end{lstlisting}
    \end{itemize}

    \subsubsection{Cron}\label{cron}

    \begin{itemize}
    \item
      Remove user cron jobs

\begin{lstlisting}
crontab -e -u sysadmin
\end{lstlisting}
    \item
      Restart cron service to apply changes

\begin{lstlisting}
systemctl restart cron
\end{lstlisting}
    \end{itemize}

    \subsubsection{Banner}\label{banner}

    \begin{itemize}
    \item
      Uncomment lines and add message

\begin{lstlisting}
vim /etc/gdm3/greeter.d/conf-defaults

banner-message-enable=true
banner-message-text=’<message>’
\end{lstlisting}
    \item
      Restart machine to apply changes
    \end{itemize}

    \subsubsection{Confirm Bind}\label{confirm-bind}

    \begin{itemize}
    \item
      Add these lines

\begin{lstlisting}
vim /etc/apparmor.d/usr.sbin.named

/var/log/bind9** rw,
/var/log/bing9/ rw,
\end{lstlisting}
    \item
      Restart apparmor and start bind

\begin{lstlisting}
systemctl restart apparmor
systemctl start bind9
systemctl status bind9
\end{lstlisting}
    \end{itemize}

    \subsubsection{Backup 2}\label{backup-2}

    \begin{itemize}
    \item
      Backup \passthrough{\lstinline!/etc!}

\begin{lstlisting}
cd /
tar -cf bcte etc/
\end{lstlisting}
    \end{itemize}

    \subsubsection{Firewall}\label{firewall-2}

    \begin{itemize}
    \item
      Install \passthrough{\lstinline!ufw!}

\begin{lstlisting}
apt install ufw
\end{lstlisting}
    \item
      Set firewall rules

\begin{lstlisting}
ufw status
ufw status verbose
ufw default deny incoming
ufw allow 53
ufw allow 953/tcp
ufw enable
\end{lstlisting}
    \end{itemize}

    \subsubsection{Security}\label{security}

    \begin{itemize}
    \item
      Download LinPeas

\begin{lstlisting}
wget https://github.com/peass-ng/PEASS-ng/releases/latest/download/linpeas.sh
\end{lstlisting}
    \item
      Run Script

\begin{lstlisting}
chmod +x linpeas.sh
./linpeas.sh | less -RN
\end{lstlisting}
    \end{itemize}

    \subsection{\texorpdfstring{\textbf{Ubuntu 18
    Server}}{Ubuntu 18 Server}}\label{ubuntu-srv}

    \subsubsection{Network}\label{network-1}

    \begin{itemize}
    \item
      DNS: Should be \passthrough{\lstinline!8.8.8.8!}
    \item
      Update apt repos to pull from archive mirror

\begin{lstlisting}
sed -i 's/archive.ubuntu.com/old-releases.ubuntu.org/g' /etc/apt/sources.list
sed -i 's/security.ubntu.com/old-releases.ubuntu.org/g' /etc/apt/sources.list
\end{lstlisting}
    \end{itemize}

    \subsubsection{User Admin}\label{user-admin-1}

    \begin{itemize}
    \item
      Change passwords

\begin{lstlisting}
passwd
passwd sysadmin
\end{lstlisting}
    \item
      Add backup user

\begin{lstlisting}
useradd <backup_username>
password <backup_password>
\end{lstlisting}
    \item
      Add backup user to sudo group

\begin{lstlisting}
usermod -aG sudo <backup_username>
\end{lstlisting}
    \end{itemize}

    \subsubsection{Services}\label{services-1}

    \begin{itemize}
    \item
      Nuke SSH

\begin{lstlisting}
systemctl stop sshd
apt remove openssh-server
\end{lstlisting}
    \item
      Disable vulnerable services

\begin{lstlisting}
systemctl stop apache2
systemctl disable apache2
\end{lstlisting}
    \end{itemize}

    \subsubsection{Database}\label{database-1}

    \begin{itemize}
    \item
      Backup and stop database

\begin{lstlisting}
mkdir /usr/lib64/b/

cd <backup directory>

mysqldump -p openshop_db > opensho.sql
(password just press enter)
mysqldump -p db > data.sql
mysqldump -p mysql > my.sql

systemctl stop mariadb
\end{lstlisting}
    \end{itemize}

    \subsubsection{Script}\label{script-1}

    \begin{itemize}
    \item
      Download scripts

\begin{lstlisting}
curl -O https://github.com/SUU-Cybersecurity-Club/CCDC-Scripts/releases/latest/linux-hardening.tar.xz
\end{lstlisting}
    \item
      Extract archive and enter directory

\begin{lstlisting}
tar -xpf linux-hardening.tar.xz
cd linux-hardening
\end{lstlisting}
    \item
      Mark executable

\begin{lstlisting}
chmod +x start.sh linux_wazuh_agent.sh
\end{lstlisting}
    \item
      Run Scripts

\begin{lstlisting}
./start.sh
\end{lstlisting}
    \end{itemize}

    \subsubsection{Banner}\label{banner-1}

    \begin{itemize}
    \item
      Edit ssh config

\begin{lstlisting}
vim /etc/ssh/sshd_config

banner /etc/issuse.net
\end{lstlisting}
    \item
      Edit or create \passthrough{\lstinline!/etc/issuse.net!}

\begin{lstlisting}
vim /etc/issuse.net
cp /etc/issuse.net /etc/issue
\end{lstlisting}
    \end{itemize}

    \subsubsection{Backup}\label{backup}

    \begin{itemize}
    \item
      Backup \passthrough{\lstinline!/etc!}

\begin{lstlisting}
cd /
tar -cf ettc etc/
mkdir
cp /ettc <dest>
\end{lstlisting}
    \item
      Backup \passthrough{\lstinline!/var/www!}

\begin{lstlisting}
cd /var/www
tar -cf web html/
mv web /etc/<dir>
\end{lstlisting}
    \end{itemize}

    \subsubsection{Firewall}\label{firewall-3}

    \begin{itemize}
    \item
      Shouldn't have anything scored so deny incoming traffic

\begin{lstlisting}
apt install ufw
ufw status
ufw default deny incoming
ufw enable
\end{lstlisting}
    \end{itemize}

    \subsection{\texorpdfstring{\textbf{Ubuntu
    Workstation}}{Ubuntu Workstation}}\label{ubuntu-wrk}

    \subsubsection{Networking}\label{networking}

    \begin{itemize}
    \item
      Point DNS to \passthrough{\lstinline!8.8.8.8!}
    \item
      Update apt repos to pull from archive mirror

\begin{lstlisting}
sed -i 's/archive.ubuntu.com/old-releases.ubuntu.org/g' /etc/apt/sources.list
sed -i 's/security.ubntu.com/old-releases.ubuntu.org/g' /etc/apt/sources.list
\end{lstlisting}
    \end{itemize}

    \subsubsection{User Admin}\label{user-admin-2}

    \begin{itemize}
    \item
      Change passwords

\begin{lstlisting}
passwd
passwd sysadmin
\end{lstlisting}
    \item
      Add backup user

\begin{lstlisting}
useradd <backup_username>
password <backup_password>
\end{lstlisting}
    \item
      Add backup user to sudo group

\begin{lstlisting}
usermod -aG sudo <backup_username>
\end{lstlisting}
    \end{itemize}

    \subsubsection{Services}\label{services-2}

    \begin{itemize}
    \item
      Nuke SSH

\begin{lstlisting}
apt remove openssh-server
\end{lstlisting}
    \item
      Stop and disable services

\begin{lstlisting}
systemctl stop avahi-daemon
systemctl disable avahi-daemon
systemctl stop cupsd
systemctl disable cupsd
systemctl stop redis-server
systemctl disable redis-server
\end{lstlisting}
    \end{itemize}

    \subsubsection{Banner}\label{banner-2}

    \begin{itemize}
    \item
      Uncomment lines and add message

\begin{lstlisting}
vim /etc/gdm3/greeter.d/conf-defaults

banner-message-enable=true
banner-message-text=’<message>’
\end{lstlisting}
    \item
      Restart machine to apply changes
    \end{itemize}

    \subsubsection{Script}\label{script-2}

    \begin{itemize}
    \item
      Download scripts

\begin{lstlisting}
curl -O https://github.com/SUU-Cybersecurity-Club/CCDC-Scripts/releases/latest/linux-hardening.tar.xz
\end{lstlisting}
    \item
      Extract archive and enter directory

\begin{lstlisting}
tar -xpf linux-hardening.tar.xz
cd linux-hardening
\end{lstlisting}
    \item
      Mark executable

\begin{lstlisting}
chmod +x start.sh linux_wazuh_agent.sh
\end{lstlisting}
    \item
      Run Scripts

\begin{lstlisting}
./start.sh
\end{lstlisting}
    \end{itemize}

    \subsubsection{Firewall}\label{firewall-4}

    \begin{itemize}
    \item
      Shouldn't have anything scored so deny incoming traffic

\begin{lstlisting}
apt install ufw
ufw status
ufw default deny incoming
ufw enable
\end{lstlisting}
    \end{itemize}

    \subsection{\texorpdfstring{\textbf{CentOS 7
    E-Comm}}{CentOS 7 E-Comm}}\label{centos}

    \subsubsection{Network}\label{network-2}

    \begin{itemize}
    \item
      DNS: Should be \passthrough{\lstinline!8.8.8.8!}
    \item
      Update yum repos to pull from vault mirror

\begin{lstlisting}
sed -i -E 's|mirror\.centos\.org|vault.centos.org|g; s|^#baseurl=http://vault|baseurl=http://vault|' /etc/yum.repos.d/*.repo 
\end{lstlisting}
    \end{itemize}

    \subsubsection{User Admin}\label{user-admin-3}

    \begin{itemize}
    \item
      Change passwords

\begin{lstlisting}
passwd
passwd sysadmin
\end{lstlisting}
    \item
      Add backup user

\begin{lstlisting}
useradd <backup_username>
password <backup_password>
\end{lstlisting}
    \item
      Add backup user to sudo group

\begin{lstlisting}
usermod -aG sudo <backup_username>
\end{lstlisting}
    \end{itemize}

    \subsubsection{Packages}\label{packages}

    \begin{itemize}
    \item
      Update repos

\begin{lstlisting}
sed -i -e '/^mirrorlist/d;/^#baseurl=/{s,^#,,;s,/mirror,/vault,;}' /etc/yum.repos.d/CentOS*.repo
\end{lstlisting}

      If that doesn't work, then run this
      \passthrough{\lstinline!sed -i -e '/\^mirrorlist/d;/\^baseurl=/\{s,\^\#,,;s,/mirror,/vault,;\}' /etc/yum.repos.d/CentOS*.repo!}
    \end{itemize}

    \subsubsection{Services}\label{services-3}

    \begin{itemize}
    \item
      Nuke SSH

\begin{lstlisting}
yum remove openssh-server
systemctl stop sshd
systemctl disable sshd
\end{lstlisting}
    \item
      Mess with services

\begin{lstlisting}
systemctl stop chronyd
crontab -e (remove currl)
systemctl restart cron
chmod -R 555 prestashop

rm /etc/httpd/conf.d/phpMyAdmin.conf
\end{lstlisting}
    \end{itemize}

    \subsubsection{Backup 1}\label{backup-1-1}

    \begin{itemize}
    \item
      Backup \passthrough{\lstinline!/etc!}

\begin{lstlisting}
cd /
tar -cf bcte etc/
\end{lstlisting}
    \item
      Backup webserver stuff

\begin{lstlisting}
cd /var/
tar -cf wwww www/
(While here move admin site)
mysqldump -p prestashop > p.sql
(password just press enter)
Mysqldump -p mysql > my.sql
\end{lstlisting}
    \end{itemize}

    \subsubsection{Script}\label{script-3}

    \begin{itemize}
    \item
      Download scripts

\begin{lstlisting}
curl -O https://github.com/SUU-Cybersecurity-Club/CCDC-Scripts/releases/latest/linux-hardening.tar.xz
\end{lstlisting}
    \item
      Extract archive and enter directory

\begin{lstlisting}
tar -xpf linux-hardening.tar.xz
cd linux-hardening
\end{lstlisting}
    \item
      Mark executable

\begin{lstlisting}
chmod +x start.sh linux_wazuh_agent.sh
\end{lstlisting}
    \item
      Run Scripts

\begin{lstlisting}
./start.sh
\end{lstlisting}
    \end{itemize}

    \subsubsection{Firewall}\label{firewall-5}

    \begin{itemize}
    \item
      Set firewall rules

\begin{lstlisting}
firewall-cmd --list-all-zones > fireb

firewall-cmd --set-target=DROP
firewall-cmd --add-service=http --permanent
firewall-cmd --remove-service=ssh --permanent
firewall-cmd --reload
\end{lstlisting}
    \end{itemize}

    \subsubsection{Database}\label{database-2}

    \begin{itemize}
    \item
      Update admin password
    \item
      cp settings change \passthrough{\lstinline!settings.inc.php!} file
    \item
      mv to sql file
    \item
      Update
      \passthrough{\lstinline!php     ps\_employee set passwd = md5('<cookie>more-secure-with-this') where id\_employee = 1;!}
    \end{itemize}

    \subsubsection{Backup 2}\label{backup-2-1}

    \begin{itemize}
    \item
      Backup \passthrough{\lstinline!/etc!}

\begin{lstlisting}
cd /
tar -cf ettc etc/
\end{lstlisting}
    \item
      Backup webserver stuff

\begin{lstlisting}
cd /var/
tar -cf wwww www/
mysqldump -p prestashop > p.sql
(password just press enter)
Mysqldump -p mysql > my.sql
\end{lstlisting}
    \end{itemize}

    \subsection{\texorpdfstring{\textbf{Fedora 21
    Webmail}}{Fedora 21 Webmail}}\label{fedora}

    \subsubsection{Network}\label{network-3}

    \begin{itemize}
    \item
      DNS: Should be \passthrough{\lstinline!1.1.1.1!}
    \item
      Update dnf repos to pull from archive mirror

\begin{lstlisting}
sed -i -E 's/^(metalink|mirrorlist)=/#\0/' /etc/yum.repos.d/fedora*.repo 
sed -i -E "/^\[fedora\]/,/^\[/{s|^[# ]*baseurl=.*|baseurl=https://archives.fedoraproject.org/pub/archive/fedora/linux/releases/\$releasever/Everything/\$basearch/os/}; /^\[updates\]/,/^\[/{s|^[# ]*baseurl=.*|baseurl=https://archives.fedoraproject.org/pub/archive/fedora/linux/updates/\$releasever/Everything/\$basearch/}" /etc/yum.repos.d/fedora.repo /etc/yum.repos.d/fedora-updates.repo
\end{lstlisting}
    \end{itemize}

    \subsubsection{User Admin}\label{user-admin-4}

    \begin{itemize}
    \item
      Change passwords

\begin{lstlisting}
passwd
passwd sysadmin
\end{lstlisting}
    \item
      Add backup user

\begin{lstlisting}
useradd <backup_username>
password <backup_password>
\end{lstlisting}
    \item
      Add backup user to wheel group

\begin{lstlisting}
usermod -aG wheel <backup_username>
\end{lstlisting}
    \end{itemize}

    \subsubsection{Script}\label{script-4}

    \begin{itemize}
    \item
      Download scripts

\begin{lstlisting}
curl -O https://github.com/SUU-Cybersecurity-Club/CCDC-Scripts/releases/latest/linux-hardening.tar.xz
\end{lstlisting}
    \item
      Extract archive and enter directory

\begin{lstlisting}
tar -xpf linux-hardening.tar.xz
cd linux-hardening
\end{lstlisting}
    \item
      Mark executable

\begin{lstlisting}
chmod +x start.sh linux_wazuh_agent.sh
\end{lstlisting}
    \item
      Run Scripts

\begin{lstlisting}
./start.sh
\end{lstlisting}
    \end{itemize}

    \subsubsection{Backup 1}\label{backup-1-2}

    \begin{itemize}
    \item
      Backup \passthrough{\lstinline!/etc!}

\begin{lstlisting}
cd /
tar -cf bcte etc/
\end{lstlisting}
    \item
      Backup mail, website, and db

\begin{lstlisting}
/var/mail
/var/ww/html
mysqldump -p roundcubemail > r.sql
(password just press enter)
Mysqldump -p mysql > my.sql
\end{lstlisting}
    \end{itemize}

    \subsubsection{Services}\label{services-4}

    \begin{itemize}
    \item
      Nuke SSH

\begin{lstlisting}
yum remove openssh-server
systemctl stop sshd
systemctl disable sshd
\end{lstlisting}
    \item
      Disable other services

\begin{lstlisting}
systemctl stop cockpit
systemctl disable cockpit
systemctl stop httpd
systemctl disable httpd
\end{lstlisting}
    \item
      Remove packages

\begin{lstlisting}
yum remove cockpit
\end{lstlisting}
    \end{itemize}

    \subsubsection{Security}\label{security-1}

    \begin{itemize}
    \item
      Remove apacheapache from wheel group

\begin{lstlisting}
vim /etc/group

Wheel delete the apache apache text
\end{lstlisting}
    \item
      Remove fake README

\begin{lstlisting}
chattr -i /etc/sudoers.d/README
rm /etc/sudoers.d/README
\end{lstlisting}
    \item
      Remove user \passthrough{\lstinline!system!} disable
      \passthrough{\lstinline!apache!} login

\begin{lstlisting}
vim /etc/passwd
Delete system user at the bottom
Also update apache from /bin/ to /sbin/nologin
\end{lstlisting}
    \item
      Move \passthrough{\lstinline!/root/passlist.txt!} and
      \passthrough{\lstinline!user.sh!} to safe place

\begin{lstlisting}
mv to new safe place
\end{lstlisting}
    \item
      Remove passlist from \passthrough{\lstinline!/etc/dovecot!}

\begin{lstlisting}
rm /etc/dovecoct/passlist.txt
\end{lstlisting}
    \end{itemize}

    \subsubsection{Firewall}\label{firewall-6}

    \begin{itemize}
    \item
      Make sure fallwall service is up

\begin{lstlisting}
firewall-cmd --list-all-zones > fireb
systemctl start firewalld
\end{lstlisting}
    \item
      Set firewall rules

\begin{lstlisting}
firewall-cmd --set-target=DROP --permanent
firewall-cmd --remove-service=cockpit --permanent
firewall-cmd --remove-service=ssh --permanent
firewall-cmd --remove-service=dhcpv6-client --permanent
firewall-cmd --add-service=smtp --permanent
firewall-cmd --add-service=pop3 --permanent
firewall-cmd --reload
\end{lstlisting}
    \item
      Backup firewall

\begin{lstlisting}
firewall-cmd --list-all-zones > firebafter
\end{lstlisting}
    \end{itemize}

    \subsubsection{Backup 2}\label{backup-2-2}

    \begin{itemize}
    \item
      Backup \passthrough{\lstinline!/etc!}

\begin{lstlisting}
cd /
tar -cf etc2 etc/
\end{lstlisting}
    \item
      Backup mail, website, and db

\begin{lstlisting}
/var/mail
/var/ww/html
mysqldump -p roundcubemail > r.sql
(password just press enter)
Mysqldump -p mysql > my.sql
\end{lstlisting}
    \end{itemize}

    \subsubsection{Distro Upgrade}\label{distro-upgrade}

    \textbf{22}

    \begin{itemize}
    \item
      Run

\begin{lstlisting}
systemctl enable postfix
yum install fedup
fedup --network 22
dnf system-upgrade reboot
check openssh-server and cockpit
\end{lstlisting}

      As per https://fedoramagazine.org/upgrade-fedora-21-fedora-22/
    \end{itemize}

    \textbf{23}

    \begin{itemize}
    \item
      Run

\begin{lstlisting}
dnf update
fedup --network 23
dnf system-upgrade reboot
\end{lstlisting}
    \item
      Make sure to check services and firewall
    \end{itemize}

    \subsection{\texorpdfstring{\textbf{2019
    AD/DNS/DHCP}}{2019 AD/DNS/DHCP}}\label{ad-dns-dhcp}

    \subsubsection{Network}\label{network-4}

    \begin{itemize}
    \item
      To get to network adapter settings: run -\textgreater{} ncpa.cpl
    \item
      To see default nameserver run nslookup
    \item
      If it's an IPv6 address, just turn off IPv6
    \end{itemize}

    \subsubsection{AD Users}\label{ad-users}

    \begin{itemize}
    \tightlist
    \item
      Descriptions have scoring users
    \end{itemize}

    \subsection{\texorpdfstring{\textbf{Splunk}}{Splunk}}\label{splunk}

    \subsubsection{Network}\label{network-5}

    \begin{itemize}
    \tightlist
    \item
      DNS: Should be \passthrough{\lstinline!8.8.8.8!}
    \end{itemize}

    \subsubsection{Services}\label{services-5}

    \begin{itemize}
    \item
      Nuke SSH

\begin{lstlisting}
yum remove openssh
\end{lstlisting}
    \item
      Disable other services

\begin{lstlisting}
systemctl stop cockpit
systemctl disable cockpit
\end{lstlisting}
    \end{itemize}

    \subsubsection{Backup}\label{backup-3}

    \begin{itemize}
    \tightlist
    \item
      Backup \passthrough{\lstinline!/opt!} and
      \passthrough{\lstinline!/etc!}
    \end{itemize}

    \subsubsection{Firewall}\label{firewall-7}

    \begin{itemize}
    \item
      Add ports to \passthrough{\lstinline!firewalld!}

\begin{lstlisting}
1514, 1515, 1516
5500, 443
9200 / 9300 / 9400
\end{lstlisting}
    \end{itemize}

    \subsubsection{Notes}\label{notes}

    \begin{itemize}
    \item
      Ui\_access log

\begin{lstlisting}
/opt/splunk/var/log/splunk/splunkd_ui_access.log
\end{lstlisting}
    \item
      Backup \passthrough{\lstinline!/opt!} move to
      \passthrough{\lstinline!/home!}
    \item
      Change passwords to splunk users in the web gui (localhost:8000)
    \item
      Find cve allowing remote access splunk 9.1.1
    \end{itemize}

    \section{Services}\label{services-6}

    \subsection{\texorpdfstring{\textbf{Passlist}}{Passlist}}\label{passlist}

    \begin{itemize}
    \tightlist
    \item
      CSV update user passwords on windows and linux script
    \end{itemize}

    \subsection{\texorpdfstring{\textbf{FTPS}}{FTPS}}\label{ftps}

    \begin{itemize}
    \item
      https://www.youtube.com/watch?v=ISVyGxYfAGg
    \item
      https://github.com/rhrn/docker-vsftpd/tree/master
    \end{itemize}

    \subsection{\texorpdfstring{\textbf{WAF}}{WAF}}\label{waf}

    \begin{itemize}
    \item
      Website application firewall
    \item
      https://owasp.org/www-project-coraza-web-application-firewall/\#
    \item
      https://hub.docker.com/\_/caddy
    \item
      https://github.com/jptosso/coraza-waf-docker/blob/master/Caddyfile
    \item
      https://medium.com/@jptosso/implementing-coraza-waf-with-docker-a55a995f055e
    \end{itemize}

    \subsection{\texorpdfstring{\textbf{LibreNMS}}{LibreNMS}}\label{librenms}

    \begin{itemize}
    \item
      https://docs.librenms.org/Installation/Docker/
    \item
      https://nagios-plugins.org/doc/man/check\_http.html
    \end{itemize}

    \subsubsection{Useful info}\label{useful-info}

    \begin{itemize}
    \item
      Needs to ping
    \item
      In docker container \passthrough{\lstinline!librenms!}

\begin{lstlisting}
lnms user:add --role=admin <username>
\end{lstlisting}
    \end{itemize}

    \section{Networks}\label{networks}

    \subsection{\texorpdfstring{\textbf{Palo
    Alto}}{Palo Alto}}\label{palo}

    \subsubsection{Wazuh}\label{wazuh-1}

    \begin{itemize}
    \item
      Ports needed

\begin{lstlisting}
1514, 1515, 1516
5500, 443
9200 / 9300 / 9400
\end{lstlisting}
    \end{itemize}

    \subsubsection{Rules}\label{rules}

    \begin{itemize}
    \item
      Interface mgmt disable ping telnet ssh http http ocsp
    \item
      Block 3306 port for all machines
    \item
      Block 4444 this is metasploits default
    \end{itemize}

    \subsection{\texorpdfstring{\textbf{HoneyPots}}{HoneyPots}}\label{honeypots}

    \subsubsection{Installation}\label{installation}

    \begin{itemize}
    \item
      Use pip

\begin{lstlisting}
python -m pip install honeypots
\end{lstlisting}
    \end{itemize}

    \subsubsection{Examples}\label{examples}

    \begin{itemize}
    \item
      Start honeypot

\begin{lstlisting}
sudo python3 -m honeypots --setup http:80,https:443,ftp:21,smtp:25,pop3:110,imap:143,telnet:23,mysql:3306,postgresql:5432,redis:6379,mongodb:27017 --options capture_commands >> logs.txt
\end{lstlisting}
    \item
      Redirect log file

\begin{lstlisting}
sudo python3 -m honeypots --setup ssh:22,ftp:21,imap:143,mysql:3306,pop3:110,smtp:25 --options capture_commands >> log.txt
\end{lstlisting}
    \item
      Use a config file

\begin{lstlisting}
sudo python3 -m honeypots --options capture_commands --config config.json >> log.txt
\end{lstlisting}
    \end{itemize}
\end{document}
